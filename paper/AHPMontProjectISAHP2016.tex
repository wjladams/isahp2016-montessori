\documentclass[11pt]{article}
\usepackage{fancyhdr}
\usepackage{titling}
\usepackage{natbib}
\usepackage[top=1in,bottom=1in,left=1.5in,right=1.5in]{geometry}
%\usepackage{fullpage}
\title{AHP FOR CLASS DECISIONS IN A MONTESSORI ELEMENTARY CLASS}
\pagestyle{fancy}
\lhead{}
\chead{\textit{ISAHP Article: \thetitle}}
\rhead{}
\lfoot{\textit{International Symposium on the \\ Analytic Hierarchy Process}}
\cfoot{\thepage}
\rfoot{London, U.K. \\ August 4-7, 2016}
\renewcommand{\headrulewidth}{0pt}
\renewcommand{\footrulewidth}{0pt}
\renewcommand{\maketitle}{\begin{center}{\Large \textbf{\thetitle}} \end{center}}
%%%Numbered bib section
\renewcommand{\bibsection}{\section{Key References}}
%%%%%%%%%
\begin{document}
\maketitle
\thispagestyle{fancy}
\begin{abstract}
A Montessori elementary class consists of students aged 6-12.  Part of the Montessori educational
process is giving the students responsibilities and decisions to make for their class.  In this see \citep{1:200} for more stuff.
\end{abstract}

\section{Introduction}
Provide here the context for your study. What motivated you to do this work? What is your research question or overall goal with this study? Why is this important? 

\section{Literature Review}
We do not expect an exhaustive literature review here. However, the reader would like to know which are the 3 to 5 key articles that have informed your study and also your conclusion of what we know (or don’t) about the topic and how this study will fill in that gap. 

\section{Hypotheses/Objectives}
Your study may be trying to prove something or at the very least will have a specific objective (e.g. development of a decision model for a particular problem). Make sure to list them here. The reader must be clear about the specific objectives or hypotheses in your study. 

\section{Research Design/Methodology}
This is an important part of a study. If you are developing an AHP/ANP model, it is important to explain how the model came to be. Typically, a model is based on information of secondary sources (literature review), pool of experts, user surveys, or a combination of sources. It is important that you explain how goal, criteria, alternatives, influences and judgments were obtained. In particular, if the opinion of different people were aggregated, explain how this aggregation took place. Similarly, indicate how you addressed inconsistency problems. Otherwise your proposed model will have a bad start for lack of face validity.

\section{Data/Model Analysis}
If you are developing an AHP/ANP model, make sure to show the hierarchy or network respectively. Show the judgment matrices or if the model is too big, show sample matrices. Make sure to provide the key indicators (e.g. consistency indices) to ensure your model is valid. Use the graphic to include additional information such as local weights, C.I., etc. Remember the old saying that a good drawing is worth 10,000 words!

\section{Limitations}
Given the reality of our physical world, no study is perfect. Please, indicate which are those aspects of the study you could not solve satisfactorily and that could weaken the value of your model. What you would different should you have the opportunity to start over? These considerations are important to assess the context of your model validity and also as a way to mentor colleagues interested in this stream of scholarship. 

\section{Conclusions}
Now you have the opportunity to share with the world your contributions. What can you conclude from this study? What are your theoretical and/or practical contributions? How can you be certain of your achievements (e.g. is the model being used?) What are the next studies you would propose in this line of scholarship? How does your contribution fit or differs into (or from) the current stream of scholarship. Make sure that your conclusions are based on the current study and that clearly show an important contribution to the theory and/or practice of AHP/ANP. 

\section{Key References}
List here only those 3 to 5 references that are key for the study at hand. As an example:

Mu, E. (2006). A unified framework for site selection and business forecasting using ANP. Journal of Systems Science and Systems Engineering, 15(2), 178–188.

Saaty, T.L., \& Peniwati, K. (2007). Group decision-making: Drawing out and reconciling differences. Pittsburgh, PA: RWS Publications.

Saaty, T.L., \& Shang, J.S. (2007). Group decision-making: Head-count versus intensity of preference. Socio-Economic Planning Sciences, 41, 22–37.

http://link.springer.com/article/10.1023%2FA%3A1024948630255


\bibliographystyle{apalike}
\bibliography{OurPapers}

\section{Appendices}
Occasionally, you may want to include appendices such as a survey sample (if data was collected via survey), a picture (e.g. a particular geographical site), etc. Appendices are not needed but if you believe they can improve the readability of your submission, they can be included here provided the total number of pages for your submission does not go above the limit indicated in the first page (short abstract).
\end{document}